\documentclass[draft, pdftex, a4paper, 12pt, openbib, ]{article}

\usepackage[pdftex, final, pdfstartview = FitV, linktocpage = false, breaklinks = true, ]{hyperref}
\usepackage[english]{babel}
\usepackage{float}   % Improved interface for floating objects ; add [H] option
\usepackage{amssymb}
\usepackage{amsmath}

%%%%%%%%%%%%%%%%%%%%%%%%%%%%%%%%%%%%%%%%%%%%%%%%%%%%%%%%%%%%%%%%%%%%%%%%%%%%%%%
% FONTS and ENCODING %%%%%%%%%%%%%%%%%%%%%%%%%%%%%%%%%%%%%%%%%%%%%%%%%%%%%%%%%%
%%%%%%%%%%%%%%%%%%%%%%%%%%%%%%%%%%%%%%%%%%%%%%%%%%%%%%%%%%%%%%%%%%%%%%%%%%%%%%%
\usepackage{lmodern}        % Latin Modern family of fonts
\usepackage[T1]{fontenc}    % fontenc is oriented to output, that is, what fonts to use for printing characters. 
% http://tex.stackexchange.com/questions/44694/fontenc-vs-inputenc 
% http://tex.stackexchange.com/questions/664/why-should-i-use-usepackaget1fontenc
% WHICH ONE TO CHOOSE?
% \usepackage{pslatex}
% \usepackage[varg, cmintegrals, cmbraces, ]{newtxtext,newtxmath} % libertine, uprightGreek (U.S.) or slantedGreek (ISO), 
% \usepackage{tgtermes}                
% \usepackage{txfonts}
% \usepackage{mathptmx}
% \usepackage[scaled=.90]{helvet}
% \usepackage{courier}
% \usepackage{textcomp}     % required for special glyphs
% \usepackage{bm}           % load after all math to give access to bold math
\usepackage[utf8]{inputenc} % inputenc allows the user to input accented characters directly from the keyboard; 
% utf8x : much broader but less compatible ; latin1 : old?
% http://tex.stackexchange.com/questions/44694/fontenc-vs-inputenc

% References:
% http://www.latex-community.org/forum/viewtopic.php?f=8&t=6637
% ftp://ftp.rrzn.uni-hannover.de/pub/mirror/tex-archive/info/l2tabu/english/l2tabuen.pdf
% ftp://ftp.dante.de/tex-archive/info/l2tabu/english/l2tabuen.pdf
% http://xpt.sourceforge.net/techdocs/language/latex/latex32-LaTeXAndFonts/single/
% http://thirteen-01.stat.iastate.edu/wiki/LaTeXFonts
% http://www.tex.ac.uk/tex-archive/info/beginlatex/html/chapter8.html
% http://tex.stackexchange.com/questions/56876/times-new-roman-fonts-and-maths-without-mathptmx


%%%%%%%%%%%%%%%%%%%%%%%%%%%%%%%%%%%%%%%%%%%%%%%%%%%%%%%%%%%%%%%%%%%%%%%%%%%%%%%
% LAY OUT %%%%%%%%%%%%%%%%%%%%%%%%%%%%%%%%%%%%%%%%%%%%%%%%%%%%%%%%%%%%%%%%%%%%%
%%%%%%%%%%%%%%%%%%%%%%%%%%%%%%%%%%%%%%%%%%%%%%%%%%%%%%%%%%%%%%%%%%%%%%%%%%%%%%%
%
% See: http://tex.stackexchange.com/questions/59626/nicely-force-66-characters-per-line
% (must be after pslatex, tgterms, etc...)
% \usepackage[DIV=calc]{typearea}
%
% \usepackage[cm]{fullpage} % set 'default' full page
% \usepackage{geometry}     % margins?
\usepackage{lipsum}         % to fill in with arbitrary text
\widowpenalty = 4000        % help suppress widows,  default = 4,000 (?), from 0 to 10 000 (from 300 to 1 000 recommended, 10 000 not recommended)
\clubpenalty  = 4000        % help suppress orphans, default = 4,000 (?), from 0 to 10 000 (from 300 to 1 000 recommended, 10 000 not recommended)
\usepackage[final, babel]{microtype} % many good lay-out/justification effects, see:
% texblog.net/latex-archive/layout/pdflatex-microtype/


%%%%%%%%%%%%%%%%%%%%%%%%%%%%%%%%%%%%%%%%%%%%%%%%%%%%%%%%%%%%%%%%%%%%%%%%%%%%%%%
% AMS MATH %%%%%%%%%%%%%%%%%%%%%%%%%%%%%%%%%%%%%%%%%%%%%%%%%%%%%%%%%%%%%%%%%%%%
%%%%%%%%%%%%%%%%%%%%%%%%%%%%%%%%%%%%%%%%%%%%%%%%%%%%%%%%%%%%%%%%%%%%%%%%%%%%%%%
% \usepackage{amsmath}      % loads amstext, amsbsy, amsopn but not amssymb
% equation stuff (eqref, subequations, equation, align, gather, flalign, multline, alignat, split...)
% \usepackage{amsfonts}     % may be redundant with amsmath
% \usepackage{amssymb}      % may be redundant with amsmath

\begin{document}	





%---------------------------------------------

$  {g}'(y) = N(x,y) - \frac{\partial }{\partial y}\int M(x,y)dx      \\\\$

%---------------------------------------------

%% Digital Control 160925

$   F - kx - bx' = mx''    \\\\$

$	F(s) -kX(s) - bsX(s) = ms^{2}X(s) \\\\ $


$  F(s) = (ms^{2} + bs + k ) X(s)     \\\\$


$	\frac{X(s)}{F(s)} = H(s) = \frac{1}{(ms^{2} + bs + k)} \\\\ $


$   H(s) = \frac{\frac{1}{m}}{(s^{2} + \frac{b}{m}s + \frac{k}{m})} \\\\ $


$   \lim\limits_{s \rightarrow 0} 2\cdot H(s) = \lim\limits_{s \rightarrow 0} 2\cdot \frac{\frac{1}{m}}{(s^{2} + \frac{b}{m}s + \frac{k}{m})} = \frac{2}{k} = 0.1   \\\\$ 


$   then, \  k = 20 N/m       \\\\$


$  t_{r} \approx \frac{1.8}{w_{n}} \approx 1      \\$


$  \therefore w_{n} = 1.8  \\$


$  w_{n} = \sqrt{\frac{k}{m}} = \sqrt{\frac{20}{m}}  = 1.8     \\$


$   then, \ m = 6.17 \ kg    \\$


$  2\xi w_{n} = 3.6\cdot\xi = \frac{b}{m} = \frac{b}{6.17}     \\$


$  M_{p} = 10\% \ \ \Longleftrightarrow \ \  \xi = 0.6   \\$


$  then, \ b = 13.32 \ N\cdot s/m     \\$




$   \dfrac{Y(s)}{R(s)} = H(s) = \dfrac{10}{s^{2} + 55s + 10}   \\\\$


$    E(s) = R(s) - Y(s) =  (\dfrac{1}{1+\dfrac{10}{s(s+55)}})R(s)   \\\\$



$   \therefore e_{ss} = \lim\limits_{s \rightarrow 0} s\cdot E(s) = \lim\limits_{s \rightarrow 0} s\cdot \dfrac{1}{1+\dfrac{10}{s(s+55)}} \dfrac{1}{s^2} = \dfrac{55}{10}  = 5.5 \ (550 \%)\\\\$



$  H(s) = \dfrac{8.1}{s^2 + 3s + 9}     \\\\$


$  H(s) = \dfrac{8.1 \cdot \dfrac{1}{5}(s+5)}{s^2 + 3s + 9}     \\\\\\\\\\$


$     x(t) = v_{0} cos(\theta) t  \\\\$


$   y(t) = v_{0} sin(\theta)  t - \dfrac{1}{2}gt^2    \\\\$


$     v_{0} = 5 m/s  \\\\$


$     n = \begin{bmatrix}
\ -sin(yaw)\cdot cos(pitch) \ \\  
\ sin(pitch) \ \\ 
\ -cos(yaw)\cdot cos(pitch) \ 
\end{bmatrix}  \\\\$


$    n = \begin{bmatrix}
\ n_{x} \ \\  
\ n_{y} \ \\ 
\ n_{z} \
\end{bmatrix}   \\\\$






\end{document}

